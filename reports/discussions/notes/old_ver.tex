\documentclass[10pt, a4paper]{article}
\usepackage[toc,page]{appendix}
\usepackage{indentfirst}
\usepackage{arydshln} % for dashed line in tabular configuration
\usepackage{afterpage}
\usepackage{pdflscape}

\usepackage{hyperref}
\usepackage[style=authoryear-comp, sorting=nyt, maxcitenames=2, maxbibnames=99, firstinits=true, hyperref=true, dashed=false, uniquename=false, uniquelist=false, backend=biber]{biblatex}
% maxcitenames working for inside article
% maxbibnames working for the printbibliography
% firstinits=true making the last name to the .

%% \usepackage[style=authoryear-comp, sorting=nyt, maxcitenames=2, backend=biber]{biblatex}

\renewbibmacro{in:}{}
\renewcommand\bibfont{\small}
\addbibresource{Brownian_dynamics.bib}
% \addbibresource{reference.bib}
% \addbibresource{trappe.bib}

\usepackage{amsmath,amssymb, mathrsfs}
\usepackage{xcolor, graphicx, epstopdf}
\usepackage{amsthm}
\newtheorem{thm}{Theorem}
\newtheorem{lemma}{Lemma}
\newtheorem{mydef}{Definition}
\title{Short Notes for Brownian Dynamics with Elastic Dumbbells}
\author{Gun Woo Park}
\linespread{1.3}
\addtolength{\hoffset}{-1.5cm}
\addtolength{\textwidth}{3cm}
\addtolength{\voffset}{-1.5cm}
\addtolength{\textheight}{3cm}

% From here, listing for source code

\usepackage{listings}
  \usepackage{courier}
 \lstset{
         basicstyle=\footnotesize\ttfamily, % Standardschrift
         numbers=left,               % Ort der Zeilennummern
         numberstyle=\color{brown}\tiny,          % Stil der Zeilennummern
         %stepnumber=2,               % Abstand zwischen den Zeilennummern
         numbersep=5pt,              % Abstand der Nummern zum Text
         tabsize=2,                  % Groesse von Tabs
         extendedchars=true,         %
         breaklines=true,            % Zeilen werden Umgebrochen
         keywordstyle=\color{blue},
    		frame=b,         
%        keywordstyle=[1]\textbf,    % Stil der Keywords
%        keywordstyle=[2]\textbf,    %
%        keywordstyle=[3]\textbf,    %
%        keywordstyle=[4]\textbf,   \sqrt{\sqrt{}} %
         stringstyle=\color{black}\ttfamily, % Farbe der String
         showspaces=false,           % Leerzeichen anzeigen ?
         showtabs=false,             % Tabs anzeigen ?
         xleftmargin=17pt,
         framexleftmargin=17pt,
         framexrightmargin=5pt,
         framexbottommargin=4pt,
         %backgroundcolor=\color{lightgray},
         showstringspaces=false      % Leerzeichen in Strings anzeigen ?        
 }
 \lstloadlanguages{% Check Dokumentation for further languages ...
         %[Visual]Basic
         %Pascal
         %C
         C++
         %XML
         %HTML
   %Java
 }
  %\DeclareCaptionFont{blue}{\color{blue}} 

  %\captionsetup[lstlisting]{singlelinecheck=false, labelfont={blue}, textfont={blue}}
  \usepackage{caption}
\DeclareCaptionFont{white}{\color{white}}
\DeclareCaptionFormat{listing}{\colorbox[cmyk]{0.43, 0.35, 0.35,0.01}{\parbox{\textwidth}{\hspace{15pt}#1#2#3}}}
\captionsetup[lstlisting]{format=listing,labelfont=white,textfont=white, singlelinecheck=false, margin=0pt, font={bf,footnotesize}}

\begin{document}
\maketitle 
%% \section{Coupled Oscillator}
%% \section{Langevin Equation for Rigid Body}

\section{Preface}
The aim of this study is for interpretation of supramolecular solution that provide unusual properties such as report of \textcite{Suzuki:2012gf} that related with Hydrophobically modified Ethoxylated uRethane (HEUR) solution. The detail explanation for HEUR solution should be refer following references: \textcite{Suzuki:2013kk, Uneyama:2012ge, Suzuki:2012gf}. This document is for methodological development in order to interpretate complicated system based on Brownian dynamics. 

For shortly, following is directions for evolution equation in dimensionless form:
\begin{enumerate}
\item Linear dumbbell model with simple Euler integrator and simple Wiener process: equation \eqref{eq:dimensionless_update_position}.
\item Non-linear dumbbell model: TBD
\item Rouse segments: TBD
\end{enumerate}

\section{Simple Brownian Motion for Beads with Repulsion}
\subsection{Basic Fomular}
The evolution equation is given by
\begin{equation}
\frac{\partial \mathbf{r}}{\partial t} = \frac{1}{\zeta}\left(\sum \mathbf{F}^{(r)} + \mathbf{F}^{(s)}\right),
\end{equation}
where $\mathbf{F}^{(r)}$ is repulsive force and $\mathbf{F}^{(s)}$ is random force contribution. With simple Euler approach, we can represent the given evolution equation as
\begin{equation}
\mathbf{r}(t + \delta t) = \mathbf{r}(t) + \frac{1}{\zeta}\sum\mathbf{F}^{(r)}(t)\delta t + \delta \mathbf{r},
\end{equation}
where stochastic step, $\delta \mathbf{R}$ is defined by
\begin{equation}
\delta \mathbf{r} = \frac{1}{\zeta}\int_t^{t+\delta t}\mathbf{F}^{(s)}(t')dt'.\label{eq:stochastic_step}
\end{equation}
Since $\mathbf{F}^{(s)}$ is following Gaussian white noise, that has following properties
\begin{align}
\langle \mathbf{F}^{(s)}(t)\rangle_t &= \mathbf{0}, \\
\langle \mathbf{F}^{(s)}(t)\mathbf{F}^{(s)}(t+\tau)\rangle_t &=2\zeta k_BT\delta(\tau)\mathbf{I},\label{eq:fluctuation_dissipation_noise}
\end{align}
the stochastic integration is evaluated by Wiener process \parencite{GREINER:1988cq, VandenBrule:1995fq}
\begin{equation}
  \delta\mathbf{r} \sim \sqrt{\delta t}\mathbf{R}(t),
\end{equation}
where $\mathbf{R}$ is random vector following uniform distribution with variance 1.
For convenience, let $\mathbf{r}_{ij} = \mathbf{r}_i - \mathbf{r}_j$, the repulsive force is defined by
The repulsive force is defined by
\begin{equation}
\mathbf{F}^{(r)}(\mathbf{r}_i, \mathbf{r}_j) = \mathbf{F}^{(r)}(\mathbf{r}_{ij}) = -Ck_BT\frac{1}{R_0}\left(1 - \frac{\mathbf{r}_{ij}^2}{R_0^2}\right)\hat{\mathbf{r}}_{ij} \quad\textrm{for } \lvert\mathbf{r}_{ij}\rvert < R_0,
\end{equation}
where hat denote directional vector, and $R_0$ denote expected size for micelle.

\subsection{Non-dimensionalization}
Since the given micelle size is fixed as $R_0$, it is appropriate to set characteristic length, $l_c$, as $R_0$. Then, we can define characteristic time as
\begin{equation}
t_c = \frac{\zeta R_0^2}{k_BT}.
\end{equation}
Note that
\begin{equation}
[t_c] = \frac{[\zeta][R_0^2]}{[k_BT]} = \frac{M\cdot T^{-1} L^2}{M\cdot L^2\cdot T^{-2}} = T.
\end{equation}

Recall

From the property of Gaussian noise (Eq. \eqref{eq:fluctuation_dissipation_noise}), it is appropriate to use 
% \begin{equation}
% \mathbf{F}^{(s)} = \sqrt{\frac{2\zeta k_BT}{t_c}}\tilde{\mathbf{F}}^{(s)},
% \end{equation}
% implies
\begin{align}
\delta \mathbf{r} = \frac{1}{\zeta}\sqrt{\delta t}\mathbf{R}(t) = 
\end{align}


\section{van den Brule's Approach using Brownian Dynamics}
Brownian dynamics is simplified version for Langevin equation that is based on evolution of position depends on random force that provide Brownian motion. The basic evolution equation on here refer to the \textcite{VandenBrule:1995fq} that is expressed by
\begin{equation}
\mathbf{v} = \mathbf{v}_0 + \mathbf{k}\cdot\mathbf{r} + \frac{1}{\zeta}\left(\mathbf{F}^{(c)} + \sum\mathbf{F}^{(r)} + \mathbf{F}^{(s)}\right),
\end{equation}
where $\mathbf{v}$ and $\mathbf{r}$ are velocity and position vectors for beads, $\zeta$ is friction coefficient, $\mathbf{F}^{(c)}$ is connecting force between particle, $\mathbf{F}^{(r)}$ is repulsive force between beads for preventing overlap, and $\mathbf{F}^{(s)}$ is random force contribution for providing Brownian motion. 
Combined with simple Euler approach, we have updated position as following representation:
\begin{equation}
\mathbf{r}(t + \delta t) = \mathbf{r}(t) + \mathbf{v}_0\delta t + \mathbf{k}\cdot\mathbf{r}\delta t + \frac{1}{\zeta}\left(\mathbf{F}^{(c)} + \sum\mathbf{F}^{(r)}(t)\right)\delta + \delta \mathbf{r},\label{eq:update_position}
\end{equation}
with following stochastic step, $\delta \mathbf{r}$:
\begin{equation}
\delta \mathbf{r} = \frac{1}{\zeta} \int_{t}^{t+\delta t}\mathbf{F}^{(s)}(t')dt'.\label{eq:stochastic_step}
\end{equation}

\subsection{Stochastic Integration and Wiener Process} 
Following basic assumptions for Brownian dynamics, the $\mathbf{F}^{(s)}(t)$ is white noise (the properties for white noise described on appendix \ref{appen_Wiener_process}); therefore, the $\delta \mathbf{r}$ is random vector following Wiener process:
\begin{equation}
\delta \mathbf{r} = \frac{1}{\zeta} \Delta \mathbf{W} (\delta t).
\end{equation}
The detail information for Wiener process is described in appendix \ref{appen_Wiener_process}, but here we describe the simple representation for Wiener process given by \parencite{GREINER:1988cq, VandenBrule:1995fq}
\begin{equation}
\Delta \mathbf{W}(\delta t) \sim \sqrt{\delta t}\mathbf{R}(t),\label{eq:Wiener_process}
\end{equation}
where $\mathbf{R}$ is random vector based on uniform distribution with given variance, $\sigma^2$. 
Detail formalism for this expression is described on following sections.

\subsection{Spring Potentials}
It is noteworthy that the approaches of \textcite{VandenBrule:1995fq} is using Gaussian spring between connectors, that is given by
\begin{equation}
\mathbf{F}^{(c)}(\mathbf{Q}) = H\mathbf{Q},
\end{equation}
where $H$ is spring constants and $\mathbf{Q}$ is vector between two beads. 
In order to implementation for finite extensibility for the (nonlinear) spring, we can use inverse Langevin function for the spring factor that has approximated analytic form such as Warner's approximation \parencite{HaroldR.Warner1972} and Pad\'e approximation \parencite{Cohen1991}. 
% \begin{equation}
% \mathbf{F}^{(c)}(\mathbf{Q}) = \frac{H}{3}\mathscr{L}^{-1}\left(\mathbf{\tilde{Q}}\right)\mathbf{\hat{Q}}
% \end{equation}
On this regards, the spring force represented by $(H/3)\mathscr{L}^{-1}(\tilde{\mathbf{Q}})\hat{\mathbf{Q}}$ where hat denote directional vector and tilde denote normalized vector with maximally extendable length, and the approximated inverse Langevin function gave us following form:
\begin{align}
H_W(\tilde{\mathbf{Q}}) &= H\frac{1}{1-\tilde{\mathbf{Q}}^2}, \\
H_C(\tilde{\mathbf{Q}}) &= H\frac{1-(1/3)\tilde{\mathbf{Q}}^2}{1-\tilde{\mathbf{Q}}^2},
\end{align}
where subscript $W$ and $C$ are represent Warner and Cohen, respectively.
Note that $\alpha$ and $\beta$ are scale factors that has important role to define characteristic time and length on later section, that is not clear at the moment. In addition, for small deformation both factor converge to the Gaussian spring factor $H$ described firstly.
Detail expressions will be found on later parts.

\subsection{Repulsive Potential}
In order to prevent concentration of beads, \textcite{VandenBrule:1995fq} use simple repulsive potential:
\begin{equation}
  \mathbf{F}^{(r)}(\mathbf{r}_i, \mathbf{r}_j) = \mathbf{F}^{(r)}(\mathbf{r}_i - \mathbf{r}_j) \equiv \mathbf{F}^{(r)}(\mathbf{k}) = \left\{\begin{array}{cc} -F_0 \left(1 - \frac{k}{L_{eq}}\right)\hat{\mathbf{k}} & \textrm{for } s<L_{eq}\\
      \mathbf{0} & \textrm{for } s\geq L_{eq}
    \end{array}\right.,\label{eq:repulsive_force}
\end{equation}
where $L_{eq}$ is half of equilibrium length of the dumbbells that related with the length scales described on later sections.


\section{Elastic Dumbbells}
\subsection{Implementation for Dumbbell Models}

\subsection{Linear Elastic Dumbbells}
% \subsection{Time and Length Scales for Gaussian (Hookean) Spring }
It is well-known that the time and length scales are correlated. In addition, when the system has connector between beads based on simple harmonic potential, then the effective time scale is related with the spring constants, i.e., effective length scale also function of spring constants. 
Here, simple harmonic potential is the only consideration since non-linear spring approach converge Gaussian statistics when deformation is small. However, if we consider non-linear spring, the time scales must varied as proposed scheme on \textcite{Herrchen1997}.
The entropic spring for Gaussian chain is given by
\begin{equation}
\mathbf{F}_G(\mathbf{Q}) = \frac{3k_BT}{\langle \mathbf{Q}_{eq}^2\rangle}\mathbf{Q} \left(\equiv H\mathbf{Q}\right),
\end{equation}
which implies
\begin{equation}
\langle \mathbf{Q}_{eq}^2\rangle = \frac{3k_BT}{H}.
\end{equation}
Therefore, it is appropriate to has basic length scales as
\begin{equation}
l_c = \sqrt{\langle\mathbf{Q}_{eq}^2\rangle}=\sqrt{\frac{3k_BT}{H}}.
\end{equation}
Notice that 3 represent for space dimension.
For the time scale, the typical relaxation time for Hookean (Gaussian) elastic dumbbell is used:
\begin{equation}
t_c = \frac{\zeta}{4H}.
\end{equation}
The derivation for the relaxation time refer appendix \ref{appen_relaxation_time_dumbbell}.
On this regards, the Gaussian spring force is easily expressed by
\begin{equation}
\mathbf{F}_G(\mathbf{Q}) = \frac{3k_BT}{l_c^2}l_c\tilde{\mathbf{Q}} = \sqrt{3k_BTH}\tilde{\mathbf{Q}} \equiv \sqrt{3k_BTH}\tilde{\mathbf{F}}_G(\tilde{\mathbf{Q}}).
\end{equation}
Note that from equation \eqref{eq:update_position}, we has nondimensional connection force as
\begin{equation}
\frac{\delta t}{l_c\zeta}\mathbf{F}^{(c)}(\mathbf{Q}) = \frac{t_c}{l_c\zeta}\sqrt{3k_BTH}\delta \tilde{t}\tilde{\mathbf{F}}(\tilde{\mathbf{Q}}) = \frac{\delta \tilde{t}}{4}\tilde{\mathbf{F}}(\tilde{\mathbf{Q}}).
\end{equation}
For convenience, let $F_0$ for repulsive force, equation \eqref{eq:repulsive_force}, be the same for the dimensional parameter for connecting force. Then, repulsive force acted only for the beads within half of characteristic length with following dimensionless form:
\begin{equation}
  \tilde{\mathbf{F}}^{(r)}(\tilde{\mathbf{Q}}) = - \left(1 - \frac{\tilde{Q}}{\frac{1}{2}l_c}\right)\frac{\tilde{\mathbf{Q}}}{\tilde{Q}}\quad\textrm{for }\tilde{Q}<\frac{1}{2}l_c.
\end{equation}

% \subsection{Random Forces}
From properties for Gaussian random force, appendix \ref{appen_Wiener_process}, it is appropriate to define dimensional parameters for force as following:
\begin{equation}
\mathbf{F}^{(s)} = \sqrt{\frac{2\zeta k_BT}{t_c}}\tilde{\mathbf{F}}^{(s)},
\end{equation}
implies
\begin{equation}
\Delta \mathbf{W}(\delta t) \left(\sim \sqrt{\delta t}\mathbf{R}(t)\right)= \sqrt{2\zeta k_BT t_c}\Delta\tilde{\mathbf{W}}(\delta \tilde{t}).
\end{equation}
Note that the dimensionless random vector can be defined as 
\begin{equation}
\mathbf{R}(t) = \sqrt{2\zeta k_BT}\tilde{\mathbf{R}}(\tilde{t}).
\end{equation}
On this regards, the stochastic step becomes
\begin{equation}
\delta\tilde{\mathbf{r}} = \frac{\sqrt{t_c}}{l_c}\sqrt{\frac{2k_BT }{\zeta}}\sqrt{\delta\tilde{t}}\tilde{\mathbf{R}}(\tilde{t}) = \frac{1}{6}\sqrt{\delta \tilde{t}}\tilde{\mathbf{R}}(\tilde{t}).
\end{equation}
Note that the variance for $\tilde{\mathbf{R}}(\tilde{t})$ should be 1 (with average is 0). Then, the dimensionless random vector is uniform distribution from $-\sqrt{3}$ to $\sqrt{3}$ since the variance for uniform distribution from $b$ to $a$ is $(b-a)^2/12$.

% \subsection{Evolution Equations}
Finally, we have non-dimensional form for linear dumbbell with simple Euler integrator combined with simple Wiener process as follow:
\begin{equation}
\tilde{\mathbf{r}}(\tilde{t} + \delta \tilde{t}) = \tilde{\mathbf{r}}(\tilde{t}) + \tilde{\mathbf{v}}_0\delta \tilde{t} + \tilde{\mathbf{k}}\cdot\tilde{\mathbf{r}}(\tilde{t}) \delta \tilde{t} + \frac{1}{4}\left(\tilde{\mathbf{F}}^{(c)} + \sum\tilde{\mathbf{F}}^{(r)}\right)\delta\tilde{t} + \frac{1}{6}\tilde{\mathbf{R}}(\tilde{t})\sqrt{\delta\tilde{t}}\label{eq:dimensionless_update_position}
\end{equation}

From update position, equation \eqref{eq:update_position}, we can easily non-dimensionalizable



\subsection{Non-Linear Elastic Dumbbells}
This approach will based on \textcite{Herrchen1997}.

\section{From Dumbbells to Rouse-like Segments}



\begin{appendices}
\section{Gaussian Random Distribution and Wiener Process}\label{appen_Wiener_process}
The Gaussian random noise has properties of 
\begin{align}
\langle \mathbf{F}^{(s)}(t)\rangle_t &= \mathbf{0} \\
\langle \mathbf{F}^{(s)}(t)\mathbf{F}(t+\tau)\rangle_t & = 2\zeta k_BT\delta(\tau)\mathbf{I},
\end{align}
that based on fluctuation-dissipation theorem \parencite{Kubo:1966dq}.


\section{Relaxation Time for Linear Elastic Dumbbell Models}\label{appen_relaxation_time_dumbbell}
Consider harmonic potential as
\begin{equation}
U(\mathbf{Q}) = \frac{1}{2}H\mathbf{Q}^2,
\end{equation}
where $H$ is spring constants and $\mathbf{Q}$ is vector between two beads. 
Using Green function rep
Then the autocorrelation for the $\mathbf{Q}$ becomes
\begin{equation}
\langle \mathbf{Q}(t+\tau)\cdot\mathbf{Q}(t)\rangle_t = 
\end{equation}

\section{Non-Linear Elastic Dumbbell Models}
\end{appendices}




\printbibliography

\end{document}
